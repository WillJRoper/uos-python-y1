\section*{Individual Work}

Reese, David and I (Will) need to port and edit what exists for chapters 1-5, as these already function as a good introduction to python (with a greater emphasis on functions and maybe change up some examples/exercises). We also need to make sure the exercises and assignments work in Jupyter.

I will edit this as time goes on. Anything below in black is part of the main module, \color{blue} blue \color{black} corresponds to the advanced chapters and \color{red} red \color{black} corresponds to the `Nerd Glossary' (mini descriptions not corresponding to entire chapters).

\subsection*{Will}

\begin{itemize}
	
\pybullet \texttt{numpy} and \texttt{scipy} (chapters 8 and 9).
\pybullet \color{blue}Scientific Computing linking chapter (post chapter 9).\color{black}
\pybullet \color{blue}Dictionaries and Sets.\color{black}
\pybullet \color{red} Data formats (HDF5, text, csv, FITS etc). \color{black}
	
\end{itemize}

\subsection*{Reese}

\begin{itemize}
	
\pybullet Convert from using \texttt{csv} to use \texttt{pandas} (chapter 5).
\pybullet \color{blue}Object orientated \texttt{python} (with help from Will??). \color{black}
	
\end{itemize}

\subsection*{David}

\begin{itemize}
	
\pybullet \color{blue}Fancy \texttt{matplotlib}.\color{black}
	
\end{itemize}

\subsection*{Aran}

\begin{itemize}
	
\pybullet \color{blue}Intro to C++.
\pybullet \LaTeX\ chapter including \texttt{BibTex}, \texttt{Zotero} and \texttt{Mendeley}. \color{black}
\pybullet \color{red}\texttt{ROOT}. \color{black}
	
\end{itemize}

\subsection*{Authorless or multi-author}

\begin{itemize}
	
\pybullet \color{blue}HTML (David or Reese?).
\pybullet UNIX (shell, etc).\color{black}
\pybullet \color{red} PEP 8. \color{black}
\pybullet \color{red} Brief breakdown of different IDEs (Atom -Reese, PyCharm, Spyder etc). \color{black}
\pybullet \color{red} Working with command line editors (nano, vim, emacs etc). \color{black}
\pybullet \color{red} \texttt{SeXtractor}. \color{black}
\pybullet \color{red}\texttt{XSPEC}. \color{black}
\pybullet \color{red} \texttt{julia}. \color{black} (we should just refer to the Julia course (which is free))
\pybullet \color{red} \texttt{astropy} with particular focus on how awesome \texttt{.units} and \texttt{.constants} are! \color{black}
\pybullet \color{red} `Nice' HPC use. \color{black} (Reese from Roberto comments/peeves)
\pybullet \color{red} Simple Parallisation? \color{black}
	
\end{itemize}

\newpage

\section*{General}

\begin{enumerate}
    \item Fix formatting for chapter authors.
    \item Edit authors actually involved in each chapter when known.
    \item Make sure all examples are PEP8 compliant.
\end{enumerate}

\section*{Chapter 1}

\noindent Old outcomes:
\begin{enumerate} 
\item Generate a simple flow chart.
\item Run Python on the University computers using the {\bf Jupyter} interface.
\item Interrupt a (pseudo) infinite loop.
\item Use the {\tt print} command.
\item Use Python as a simple calculator.
\item Use three different Python data types: integer, float, and complex.
\item Add comments to your Python codes.
\end{enumerate}

\noindent Things to add/change:
\begin{enumerate}
    \item Never too much emphasis on good commenting, otherwise this is a mostly good intro as it stands.
    \item Add flow chart example.
    \item Exit Verbatim.
    \item Generally clean up formatting due to porting.
    \item Change iPython ouput to jupyter style.
\end{enumerate}

\section*{Chapter 2}

\noindent Old outcomes:
\begin{enumerate}
\item still do everything described in the script for Lab Session 1,
\item add comments to your scripts,
\item manipulate strings,
\item use boolean's,
\item use tuples and lists,
\item use the {\tt input} command,
\item use indentation and colons.
\end{enumerate}

\noindent Things to add/change:

\section*{Chapter 3}

\noindent Old outcomes:
\begin{enumerate}
\item understand the use of tab indents and colons,
\item use {\tt if}, {\tt while}, {\tt for} statements,
\item use loops,
\item write basic codes,
\item use the modulo (\%) operator,
\item use the {\tt range} function,
\item use and define your own {\bf functions},
\item use the {\tt lambda} statement (optional),
\item use the dictionary data type (optional).
\end{enumerate}

\noindent Things to add/change:

\section*{Chapter 5}

\noindent Old outcomes:
\begin{enumerate}
\item Know what Python modules are and what they are for. 
\item Know how to load Python modules.
\item Know how to find out what functions are inside a given module and how to get help with those functions.
\item Be familiar with the {\tt pylab} module (including the {\tt arange} function).
\item Be familiar with the the {\tt random} module.
\item Be able to make basic level plots.
\end{enumerate}

\noindent Things to add/change:
\begin{enumerate}
    \item Obliterate any reference to pylab with as much fire as it takes.
\end{enumerate}

\section*{Chapter 6}

\noindent Old outcomes:
\begin{enumerate}
\item Be familiar with the the {\tt csv} module.
\item Be able to make intermediate level plots (including adding error bars).
\item Be able to carry out basic fitting of data to models.
\end{enumerate}

\noindent Things to add/change:

\begin{enumerate}
    \item Switch to pandas.
\end{enumerate}

\section*{Chapter 8}

\noindent Old outcomes:
\begin{enumerate}
\item Be able to make n-dimensional arrays
\item Be able to manipulate vectors and n-dimensional arrays
\item Be able to carry out matrix and vector calculations
\item Be able to plot arrays (i.e. 2D shapes).
\end{enumerate}

\noindent Things to add/change:

\section*{Chapter 9}

\noindent Old outcomes:
\begin{enumerate}
\item Be able to save image and text files.
\item Be able to make histograms with Python.
\item Be able to generate simple webpages with HTML and Python.
\end{enumerate}

\noindent Things to add/change:

\section*{Glossary}

\noindent Need to:
\begin{enumerate}
    \item Change formatting to listings.
    \item Generally fix all the latex bugs...
    \item Add anything now relevant given new module.
    \item Check Python 3 compatibility.
\end{enumerate}

\section*{Work Done/Changelog}

Update here when you have done something so pay can be attributed properly.

\subsection*{Chapter 1}

\begin{itemize}
    \pybullet \will{Ported from old version.}
    \pybullet \will{Minor bug fixing.}
    \pybullet \will{Converted all examples to lstlisting PY environments.}
    \pybullet \will{Removed incorrect python-2 text.}
    \pybullet \will{Added a section on commenting to introduce it as early as possible.}
    \pybullet \will{Added some advanced red boxes.}
\end{itemize}

\subsection*{Chapter 2}

\begin{itemize}
    \pybullet \will{Ported from old version.}
    \pybullet \will{Minor bug fixing.}
    \pybullet \will{Converted all examples to lstlisting PY environments.}
    \pybullet \will{Removed commenting subsection since now in Chapter 1.}
    \pybullet \will{Added stackoverflow.com to the preamble.}
    \pybullet \will{Added advanced section on the is keyword and moved the copy stuff to within it}
\end{itemize}

\subsection*{Chapter 3}

\begin{itemize}
    \pybullet \will{Ported from old version.}
    \pybullet \will{Minor bug fixing.}
    \pybullet \will{Converted all examples to lstlisting PY environments.}
    \pybullet \will{Converted to python 3 where necessary.}
    \pybullet \will{Added extra context around examples.}
    \pybullet \will{Made dictionary a proper section rather than optional.}
\end{itemize}

\subsection*{Chapter 4}

\begin{itemize}
    \pybullet 
\end{itemize}

\subsection*{Chapter 5}

\begin{itemize}
    \pybullet \will{Ported from old version.}
    \pybullet \will{Minor bug fixing.}
\end{itemize}

\subsection*{Chapter 6}

\begin{itemize}
    \pybullet 
\end{itemize}

\subsection*{Chapter 7}

\begin{itemize}
    \pybullet 
\end{itemize}

\subsection*{Chapter 8}

\begin{itemize}
    \pybullet 
\end{itemize}

\subsection*{Chapter 9}

\begin{itemize}
    \pybullet 
\end{itemize}

\subsection*{Chapter 10}

\begin{itemize}
    \pybullet 
\end{itemize}

\subsection*{Chapter 11}

\begin{itemize}
    \pybullet 
\end{itemize}

\subsection*{Introduction to C$++$}

\begin{itemize}
    \pybullet \aran{First draft written and encorporated.}
\end{itemize}

\subsection*{Nerd Glossary}

\begin{itemize}
    \pybullet 
\end{itemize}

