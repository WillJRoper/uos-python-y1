\section{Making C++ files and compilation}
The first thing you will notice about C/C++ is that you can't simply run the files that you write.
You'll need to compile the text file you write (this generates an executable) and then you can run the executable.
Before all of this we need to check that there is a compiler installed on your computer. 
You can do this by typing the following command into your terminal.

\lstset{style=mystyle}
\begin{lstlisting}{language=shell}
$ which g++
\end{lstlisting}
If you get a line back that looks like this
\begin{lstlisting}{language=shell}
/usr/bin/g++
\end{lstlisting}
Then you have the \texttt{g++} compiler installed. If not, install it.
A quick google search will tell you exactly how to do this.

Now that we have the compiler working we can write our first \texttt{C++} program.
In the terminal make a new directory and navigate into it.
For example:
\begin{lstlisting}{language=shell}
$ cd Desktop/
$ mkdir IntroToCPP
$ cd IntroToCPP/
\end{lstlisting}
To create a new \texttt{.cxx} file we can use the \texttt{touch} command in linux as follows.
\begin{lstlisting}{language=shell}
$ touch first.cxx
\end{lstlisting}
Now you can open \texttt{first.cxx} with your preferred text editor.
Once you have opened it you can write the following code.
\lstset{style=mystyle}
\begin{lstlisting}
int main() {
	return 0;
}
\end{lstlisting}
Before we compile this there are some important things to take note of. 
Firstly, this is a function and it is specifically called "'main".
In \texttt{C++} you have to have a main function for the code to work.
Secondly, there is a semi-colon at the end of the line of code.
\texttt{C++} does not see new lines, therefore you have to physically denote them with semi-colons.
The third feature is that the contents of the function is surrounded by curly braces. 
Instead of indentation \texttt{C++} requires the active portion of any function to be in curly braces.
Lastly, every function in \texttt{C++} as to return something, in this case we are returning the value zero.

Now we can compile out first \texttt{C++} program.
To do so go into your terminal and enter the following command.
\begin{lstlisting}{language=shell}
$ g++ first.cxx -o first
\end{lstlisting}
What we are going here is telling the \texttt{g++} compiler to compile out \texttt{.cxx} file.
Giving the input flag \texttt{-o} tells \texttt{g++} that the word that comes next is the name we want to give the executable we're about to make.  
If you hit enter, hopefully you don't get any compilation errors. 
In order to run the executable you've just made you enter the following.
\begin{lstlisting}{language=shell}
$ ./first
\end{lstlisting}
Hit enter and nothing should happen, however this is because we've not told the program to do anything, just yet that is.

\section{Data types and modules}
\texttt{C++} is what is referred to as a strongly typed language. 
What this means is that you have to define the type of your variables before you can do anything with them.
Furthermore, you cannot perform operations on variables if they have different types, unless you change their type before doing what you're trying to do.
Some examples of data types are
\begin{itemize}
\item int: for integers,
\item float: floating point numbers,
\item double: similar to a float, but with more decimal places,
\item long: similar to and int, but longer,
\item bool: for storing boolean variables (true, false),
\item there are many, many more.
\end{itemize}

Another feature of \texttt{C++}  is that it is, by construction, a very modular language.
For \texttt{C++} there are numerous libraries, with even more numerous modules, all containing their own data types and interesting functions.
Within the context of this document we will mainly consider the \texttt{C++} standard library, within which you can find things like vectors, strings, mathematical operators and many more. 
In order to include any of these modules in the code you write - for example, if I want to use the vector module - you will need to put something like
\begin{lstlisting}{style=mystyle}
#include <vector>
\end{lstlisting}
at the top of your script.
We will discuss vectors and other features of the \texttt{C++}  standard library later in this document.

\section{Terminal inputs and outputs}
A common part of any piece of code is printing information to the terminal window.
This is handled by a module in the \texttt{C++} standard library, called the input output stream.
With this we can now write out first \texttt{C++} script that actually does something.
In a \texttt{.cxx} file you can write the following code 
\lstset{style=mystyle}
\begin{lstlisting}
#include <iostream>

int main() {
	std::cout << "Hello, world" << std::endl;
	return 0;
}
\end{lstlisting}
So, what have we done here?
First we included the input output stream, allowing us to be able to use it.
Then we called the function \texttt{cout} which means console output (the equivalent of \texttt{print()} in Python). 
This is preceded by \texttt{std::}. 
\texttt{std} is what is called a namespace, so what this means is from the standard library get this function. 
It is left to the student to research namespaces, however for this course we are going to focus on the standard library, and so we will only be using \texttt{std::}.   
At the end of the line we are using \texttt{std::endl} which is telling the computer to start a new line at the end of the print statement. 

Compile this as you did before and you should get the following output.
\begin{lstlisting}{language=shell}
$ g++ first.cxx -o first
$ ./first
Hello, world!
$
\end{lstlisting}
Congratulations! 
You've just made your first piece of \texttt{C++} code that actually does something.

Another interesting feature you can add to your code is the ability to enter things into it from the terminal while it is running.
For this we must use the console input command.
As an example, we are going to build a piece of code that takes in one integer and returns double that integer.
To do this we are going to have to use what we know about console outputs, variable definitions and console inputs.
The code will look as follows.
\begin{lstlisting}{style=mystyle}
#include <iostream>

int main() {
  int number;
  std::cout << "Please enter a number: ";
  std::cin >> number;
  std::cout << std::endl;
  
  std::cout << "Twice number you entered was: " << 2 * number << std::endl;
  return 0;
}
\end{lstlisting}
So, what have we done here?
We have to define a variable, in this case called "\texttt{number}", and we define it to be an integer.
The next line is a console output statement as you have seen before.
Now, using \texttt{std::cin} we can perform a console input. 
What this line means is: "whatever is written into the terminal, funnel that value into the variable 'number'".
It is very important here to notice the direction of the double chevrons!
At the end we are printing a short string, followed by our input number multiplied by two.
Try to compile this and run it. 
Notice what happens if you don't enter an integer into the terminal when it asks you to. \\
\\
\noindent Question: what does the computer think you're giving it if you enter a word or a non-integer value when it asks for an integer?

\section{Conditional statements}
One of the most important features of a piece of code is for it to be able to branch in one direction or another depending on a certain condition.
This allows for the user to as a question and for the computer to respond depending on what it does or doesn't want to see.
The type of function we are describing now is the \texttt{if} funciton.
Similarly to Python, \texttt{if} statements require one or more argument that has to be satisfied in order to execute what is within the function.
This would look as follows.
\begin{lstlisting}{style=mystyle}
if (my_condition) {
	perform_this_action;
}
\end{lstlisting}
Let's illustrate the use of this function with some example code.
Suppose we want the computer to tell us if a number we enter into the terminal is a multiple of 7.
You can have a go at this yourself or follow the script below.
\begin{lstlisting}{style=mystyle}
#include <iostream>

int main() {
  
  int number;
  std::cout << "Please enter a number: ";
  std::cin >> number;
  std::cout << std::endl;
  
  if (number%7 == 0) {
    std::cout << number << " is a multiple of 7" << std::endl;
  }
  else {
    std::cout << number << " is not a multiple of 7" << std::endl;
  }
  return 0;
}
\end{lstlisting}
This should appear to be similar to what you have seen before.
The only change is the \texttt{if} statement at the end.
What we are asking there is: "if the number we input modulus 7 is equal to zero print a positive output statement. Otherwise (else) print the negative output statement".

Another example of this kind of technique comes in the form of a \texttt{switch} statement.
These are similar to the functional form of an \texttt{if} statement, however the internal definitions are very different.
Suppose now we want to see if an inputted number is odd or even.
We can do this with a switch statement as follows.
\begin{lstlisting}{style=mystyle}
#include <iostream>

int main() {
  
  int number;
  std::cout << "Please enter a number: ";
  std::cin >> number;
  std::cout << std::endl;
  
  switch(number%2) {
    case 0:
      std::cout << "Even" << std::endl;
      break;
    case 1:
      std::cout << "Odd" << std::endl;
      break;
  }
  return 0;
}
\end{lstlisting}
There are numerous uses for \texttt{switch} statements as they have versatilities you don't get so conveniently with if statements. 
One we will go over later is using them for declaration of input flags, a very useful tool to use when you don't want to compile your files over and over again.

\section{Loops}
As in Python, loops are very useful when you want to perform an action many times over some value of iterations.
This type of practicality is most easily realised in the form of a \texttt{for} loop. 
\texttt{while} loops also exist in \texttt{C++}, however they are syntactically the same as in Python.
In \texttt{C++} you have to be very careful to define your loop accurately.
The general form for a \texttt{for} loop is as follows.
\begin{lstlisting}{style=mystyle}
for (start point; end point; incitement) {

}
\end{lstlisting}
One has to be careful to define the variable and the data type of the variable within the declaration of the loop.
Below is an example of a loop that will print the numbers from 1 to 10.
\begin{lstlisting}{style=mystyle}
#include <iostream>

int main() {
  
  for (int i=1; i<11; ++i) {
    std::cout << i << std::endl;
  }
  
  return 0;
}
\end{lstlisting}
Notice here that I have declared the variable \texttt{i} is an integer running from 1 to 10.
For convenience I have set the increment using \texttt{++i}.
This is a short hand for "add one to the value of i".
\\
\noindent Challenge: write a program that prints all of the numbers between two user inputted numbers in one column, and whether it is a multiple of 3, 7, and both in a column next to it.
\\
\noindent Solution:
\begin{lstlisting}{style=mystyle}
#include <iostream>

int main() {
  
  int number1;
  int number2;
  
  std::cout << "Value for number 1: ";
  std::cin  >> number1;
  std::cout << "Value for number 2: ";
  std::cin  >> number2;
  
  for (int i=number1; i<number2; ++i) {
    std::cout << i << "\t";
    if (i%3 == 0 && i%7 == 0) {
      std::cout << "both" << std::endl;
    }
    else if (i%3 == 0 && i%7 != 0) {
      std::cout << "only 3" << std::endl;
    }
    else if (i%3 != 0 && i%7 == 0) {
      std::cout << "only 7" << std::endl;
    }
    else {
      std::cout << "neither" << std::endl;
    }
  }
  return 0;
}
\end{lstlisting}

\section{Strings and chars}
It is often nice to not only have numerical outputs and manipulation, but to also have words too.
In \texttt{C++} there are two ways to achieve this: \texttt{string}, imported from the standard library, or the slightly more complicated \texttt{char}.

First we will discuss the \texttt{string} module.
As before we need to include it in in out code before we are able to use them.
\begin{lstlisting}{style=mystyle}
#include <string>
\end{lstlisting}
Now we will be able to define variables that store words!
Using what you've learned already write a program that asks for the user's first name and responds with a friendly greeting.
\\
Solution:
\begin{lstlisting}{style=mystyle}
#include <iostream>
#include <string>

int main() {
  
  std::string name;
  
  std::cout << "What is your first name: ";
  std::cin  >> name;
  
  std::cout << "Hello, " << name << std::endl;
  return 0;
  
}
\end{lstlisting}
See what happens if you try and put more than one word in the user input.
Why do you think this happens?
When you define a \texttt{std::string} in the text, i.e.
\begin{lstlisting}{style=mystyle}
std::string FullName = "A. Kathy Romer";
\end{lstlisting}
it will see the blank spaces as part of the singular string because they are within the quotation marks.
However, when you are using a command line input a blank space denotes the separation between two objects, whether that be numbers, strings or anything else.
So, how do we get around this?
Well, it's a bit complicated so it will be discussed more deeply further into this course, but the following code will allow for multiple words in the input.
\begin{lstlisting}{style=mystyle}
#include <iostream>

int main() {
  
  char name[100];
  
  std::cout << "What is your name: ";
  std::cin.getline(name, sizeof(name));
  std::cout << "Hello, " << name << std::endl;
  return 0;
}
\end{lstlisting}
See, that's definitely more complicated.
















\section{Challenges}
