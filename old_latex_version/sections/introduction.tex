\section*{Preamble}
The \texttt{Python} programming language is ubiquitous in science, especially in Physics. It is an extremely powerful tool which can be utilised for everything from making simple games to simulating the evolution of the Universe. In fact, `Big Tech' companies such as Google and Facebook use \texttt{Python} behind the scenes of many services you use daily. 

Whether or not you have programmed before this module will give you the tools you need to perform research, complete assignments and most importantly will give you marketable skills that can help you get a job after your degree. If you're anything like the authors of this document it will also have the unintended consequence of giving you an \textbf{unhealthy coding addiction}, which will be impossible to kick.

\section*{Module Structure}

This module is run in tandem with Physics In Practice (PIP) but don't be fooled, it is just as important as many of the full fat modules you'll do in your degree. As such, it is structured in much the same way as your other modules with teaching, assignments, and exercises within the lab book. The main difference is in \textbf{how} the material is taught; rather than traditional lectures you'll take part in lab sessions and work through this document chapter by chapter. These lab sessions, starting with a very short introductory talk, will give you direct contact with the AT and the lecturers to help you with any problems.
% In the light of the current situation this is not the case and instead we will be running three/four hour long sessions a week via zoom in which at least 1 of us, if not more will be present to offer help and explain concepts.

In addition to these sessions we will record and upload any relevant short lectures as videos to canvas to be digested at anytime. We also would like to draw your attention to the discussions tab on canvas, here you can post questions and one of us will get back to you as soon as we can or we can plan a full explanation for one of the workshops.

Each chapter in this document is typically a single lab session; this translates to a single chapter per week. The exceptions to this rule are weeks 4, 7, and 10 in which you will have an opportunity to work on assignments.
The first (week 4) assignment is formative and will be peer marked.
We will give you further details on this marking in the introduction sessions.
The second two assignments will be tutor marked and contribute equally to the
portfolio mark, worth 30\% of the PIP module mark.
% there will be competency tests. These competency tests enable us to gauge how everyone is progressing in a formative manner and will also lead to a competency certificate for you, showing proof of your capabilities. They will be peer marked and the results entered through a link that will be provided on canvas. 

We will repeat this again and again: do not take the low weighting of this sub-module as a reflection of it's importance, \texttt{Python} is an essential skill. Failure to engage now \textbf{will} lead to troubles down the road when the weightings are far higher and contributory. We have seen it before and, sadly, will undoubtedly see it again where a student hits a \textbf{Python} related problem in a later module which was discussed during this module. 

\section*{Lab book}
You will find the entirety of this lab book uploaded to canvas (any updates to the content will be uploaded when necessary) in case you wish to move through the course faster than planned. However, not everyone enjoys coding as much as we do, \textbf{so do not worry if you find yourself struggling a little}, please make sure to ask the lecturers or AT for help if you need it. Programming can be daunting for the uninitiated and it can take time to start thinking like a programmer, but when you do your increased understanding of logic will help you throughout your whole degree/life.

Each chapter will begin with a list of learning outcomes for that section followed by the content containing some worked examples. At the end of each section, there will be some exercises for you to complete to check your understanding of the chapter (remember to ask the ATs for help if you're stuck, they will \textbf{likely} have made all the same mistakes).

\begin{tcolorbox}[colback=red!5!white,colframe=red!75!black]

Throughout this lab book you will find boxes like this. These contain information which is beyond the scope of this module but will be important should you decide to continue into computationally heavy research or a career using \texttt{Python}. They will be useful for future reference, or are just interesting, but don't be intimidated if you don't immediately understand the content contained within them.
	
\end{tcolorbox}

%After the main body of the lab book you will find chapters that take a look at some more advances features of \texttt{Python} and scientific computing as a whole. Following the advanced chapters is an affectionately named `Nerd's Glossary', containing small sections detailing niche concepts which we have deemed `interesting' and useful for reference in the future. Should you come up with anything we've omitted that you think should be included, do let us know.

\section*{Assignments}
There will be three assignments, the first of which is unweighted and will be peer marked, and they will be marked on:

\begin{itemize}
    \item The final answers you have.
    \item Coding style (including comments!!! More on this in Chap.\ref{chap:jupyter}).
    \item How logical and efficient your approach is.
\end{itemize}

\noindent The assignments will be uploaded to Canvas and should be submitted by the deadlines in weeks 4, 7 and 10. These assignments, like all others, should be completed and submitted on your own, containing only your own work (a reminder about collusion and plagiarism is provided in section \ref{subsec:collusion}). % These assignments are in addition to the competency tests.

\section*{Cautionary Note}

This module has been redesigned to use \texttt{Python-3}. This is the newest version of  \texttt{Python} (replacing \texttt{Python-2}) and the version that's installed on the University system. You may come across instances of \texttt{Python-2} throughout your degree (and beyond) since it is still widely used but official support has now ended. There is little difference between the two versions on the surface, so don't worry about this too much. The main differences are the syntax of the \texttt{print} function and how the code behaves when dividing one integer by another.

\begin{tcolorbox}[colback=red!5!white,colframe=red!75!black]
The \texttt{print} function (as you may have guessed), prints a value to the screen.
Should you be interested, the exact differences between the \texttt{print} function and integer division in each  \texttt{Python} version are as follows:

\noindent\textbf{Python-2}
\begin{lstlisting}[style=PY]
In [1]: print 'Hello World'  # print statement, notice the lack of ()
        1 / 2  # integer division
        1.0 / 2.0  # float division
\end{lstlisting}
\begin{lstlisting}[style=PY, backgroundcolor=\color{white}]
Out[1]: 'Hello World'
        0  # 0 since 1 is not divisible by 2
        0.5  # float division gives the expected result of a half
\end{lstlisting}

\noindent\textbf{Python-3}
\begin{lstlisting}[style=PY]
In [2]: print('Hello World')  # print function, notice the parentheses
        1 / 2  # float division is automatically assumed in\texttt{Python-3}
        1 // 2  # integer division now has a different operator
\end{lstlisting}
\begin{lstlisting}[style=PY, backgroundcolor=\color{white}]
Out[2]: 'Hello World'
        0.5  # float division
        0  # integer division
\end{lstlisting}
\end{tcolorbox}



